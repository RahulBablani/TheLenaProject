\documentclass[12pt, titlepage]{article}

\usepackage{booktabs}
\usepackage{tabularx}
\usepackage{hyperref}
\hypersetup{
    colorlinks,
    citecolor=black,
    filecolor=black,
    linkcolor=red,
    urlcolor=blue
}
\usepackage[round]{natbib}

\title{SE 3XA3: Development Plan\\Title of Project}

\author{Team \#, Team Name
		\\ Student 1 name and macid
		\\ Student 2 name and macid
		\\ Student 3 name and macid
}

\date{\today}

%% Comments

\usepackage{color}

\newif\ifcomments\commentstrue

\ifcomments
\newcommand{\authornote}[3]{\textcolor{#1}{[#3 ---#2]}}
\newcommand{\todo}[1]{\textcolor{red}{[TODO: #1]}}
\else
\newcommand{\authornote}[3]{}
\newcommand{\todo}[1]{}
\fi

\newcommand{\wss}[1]{\authornote{blue}{SS}{#1}}
\newcommand{\ds}[1]{\authornote{red}{DS}{#1}}
\newcommand{\mj}[1]{\authornote{red}{MSN}{#1}}
\newcommand{\cm}[1]{\authornote{red}{CM}{#1}}
\newcommand{\mh}[1]{\authornote{red}{MH}{#1}}

% team members should be added for each team, like the following
% all comments left by the TAs or the instructor should be addressed
% by a corresponding comment from the Team

\newcommand{\tm}[1]{\authornote{magenta}{Team}{#1}}


\begin{document}

\maketitle

\pagenumbering{roman}
\tableofcontents
\listoftables
\listoffigures

\begin{table}[bp]
\caption{\bf Revision History}
\begin{tabularx}{\textwidth}{p{3cm}p{2cm}X}
\toprule {\bf Date} & {\bf Version} & {\bf Notes}\\
\midrule
Date 1 & 1.0 & Notes\\
Date 2 & 1.1 & Notes\\
\bottomrule
\end{tabularx}
\end{table}

\newpage

\pagenumbering{arabic}

This document describes the requirements for ....  The template for the Software
Requirements Specification (SRS) is a subset of the Volere
template~\citep{RobertsonAndRobertson2012}.  If you make further modifications
to the template, you should explicity state what modifications were made.

\section{Project Drivers}

\subsection{The Purpose of the Project}

\subsection{The Stakeholders}

\subsubsection{The Client}

\subsubsection{The Customers}

\subsubsection{Other Stakeholders}

\subsection{Mandated Constraints}

\subsection{Naming Conventions and Terminology}

\subsection{Relevant Facts and Assumptions}

User characteristics should go under assumptions.

\section{Functional Requirements}

\subsection{The Scope of the Work and the Product}

\subsubsection{The Context of the Work}

\subsubsection{Work Partitioning}

\subsubsection{Individual Product Use Cases}

\subsection{Functional Requirements}

\section{Non-functional Requirements}

\subsection{Look and Feel Requirements}
\begin{itemize}
\item This application’s user interface will be visually appealing and will operate seamlessly on any operating system.
\item The application will take a new approach on the way users edit and filter their photos, they will be able to achieve new and better results to attain the perfect image. 
\end{itemize}

\subsection{Usability and Humanity Requirements}
\begin{itemize}
\item The software will have a simple UI that will allow for any individual no matter their age or familiarity with image processing to be able to use the application.
\item There will be no manual or guide on how to use the application, therefore The UI must be extremely intuitive to result in a plane learning curve.
\item The software will use a standard coding style and will be well documented for anyone interested in examining the code.
\end{itemize}

\subsection{Performance Requirements}
Speed
\begin{itemize}
\item The app should take no longer than 3 seconds to load.
\item Users will be able to upload an image in a maximum of 2 seconds.
\item All interaction between a user and the UI will be instantaneous. Response times should not exceed ½ a second to complete any action. This includes all the different filters that are available.
\end{itemize}
Reliability
\begin{itemize}
\item For now we plan to release on desktop computer as this is a java application. Due to the nature of not actually needing a wireless connection nor connection to any server, the user can use this app freely, and whenever they see fit.
\item After taking the appropriate measures to accommodate for any risks that may arise during the use of the application, the program should not crash but instead send an error message if misused.
\end{itemize}
Quality
\begin{itemize}
\item Through no point in the image processing (including addition of filters and saving the new image) will the resolution be affected from its original value nor will any form of distortion occur.
\end{itemize}
Capacity
\begin{itemize}
\item Due to the local nature of the program and its independency from any sever there is no limit to the amount of people that may be processing there images.
\item Storage is based solely on the limitations of each individual user. The application can support processing of an unlimited number of photos.
\end{itemize}

\subsection{Operational and Environmental Requirements}
Physical Environment
\begin{itemize}
\item The application will be available to anyone wishing to process an image.
\end{itemize}
Technological Environment
\begin{itemize}
\item The software will run on any desktop computer and is supported on all operating systems.
\item At the moment any image size will be accepted as long as the file format is either a PNG or a JPG and will be returned as a PNG or JPG.
\item All app data and images will be stored within each users local hard drive or any other personal storage device.
\end{itemize}

\subsection{Maintainability and Support Requirements}
\begin{itemize}
\item The program will be continuously updated to fix any bugs and add more functionality.
\item New patches will be released whenever we have made any major updates or fixes so users can experience the best software we have to offer.
\item Proper documentation will be maintained as the software expands and new code is added.
\item To broaden the scope of the project we plan to start accepting more file types as well as introduce video processing capabilities.
\item The program will perform on any desktop but will also be portable, meaning it can be stored on a USB or external hard drive and run from there.
\end{itemize}

\subsection{Security Requirements}
\begin{itemize}
\item Using correct segregation of the design, we can achieve information hiding thus protecting us from potential intellectual property theft.
\end{itemize}

\subsection{Cultural Requirements}
\begin{itemize}
\item Filters will not resemble any offensive, profane or non-ethical ideologies.
\item The application will include filters which parallel current pop culture.
\end{itemize}

\subsection{Legal Requirements}
\begin{itemize}
\item Throughout the documentation we reference the use of The Marvin Framework. Our citation legalizes the use of their plugins.
\item No legal documentation is needed in terms of the images themselves as the application is only responsible for the processing of images not the supply of them.
\item Include am EULA to outline how users can use the software as well as to protect from fraud and piracy.
\end{itemize}

\subsection{Health and Safety Requirements}

This section is not in the original Volere template, but health and safety are
issues that should be considered for every engineering project.

\section{Project Issues}

\subsection{Open Issues}

\subsection{Off-the-Shelf Solutions}

\subsection{New Problems}

\subsection{Tasks}

\subsection{Migration to the New Product}

\subsection{Risks}

\subsection{Costs}

\subsection{User Documentation and Training}

\subsection{Waiting Room}

\subsection{Ideas for Solutions}

\bibliographystyle{plainnat}

\bibliography{SRS}

\newpage

\section{Appendix}

This section has been added to the Volere template.  This is where you can place
additional information.

\subsection{Symbolic Parameters}

The definition of the requirements will likely call for SYMBOLIC\_CONSTANTS.
Their values are defined in this section for easy maintenance.


\end{document}