\documentclass[12pt, titlepage]{article}

\usepackage{booktabs}
\usepackage{tabularx}
\usepackage{hyperref}
\usepackage{graphicx}
\usepackage{tabu}
\usepackage{color}
\usepackage{indentfirst}
\usepackage{multirow}

\hypersetup{
    colorlinks,
    citecolor=black,
    filecolor=black,
    linkcolor=black,
    urlcolor=blue
}
\usepackage[round]{natbib}

\title{SE 3XA3: Development Plan\\Image Processing Application}

\author{Team 1, Team NAR
		\\ Rahul Bablani - bablanr
		\\ Abeed Alibhai - alibhaa
		\\ Nezar Dimitri - dimitn
}

\date{\today}


\begin{document}

\maketitle

\pagenumbering{roman}
\tableofcontents
\listoftables
\listoffigures

\begin{table}[htbp]
\caption{\bf Revision History}
\begin{tabularx}{\textwidth}{p{3cm}p{2cm}X}
\toprule {\bf Date} & {\bf Version} & {\bf Notes}\\
\midrule
Oct 6, 2016 & 0 & Rev0 - Split up parts and started documentation.\\
Oct 11, 2016 & 0.1 & Rev0 - Finished separate parts and compiled document.\\
Dec 7, 2016 & 1 & Rev1 - 
\begin{itemize}
	\item Added professor to stakeholders
	\item "Context of the Work" explained
	\item Eye strain added to safety
	\item Added upcoming features to waiting room
\end{itemize}


\end{tabularx}
\end{table}

\clearpage

\pagenumbering{arabic}

\section{Project Drivers}

This section provides information about the project drivers within our project.

\subsection{The Purpose of the Project}

The purpose of this project, is to re-implement the open source image processing project \textit{Marvin Frameworks}. This Java framework allows clients to upload and process normal photos, into elegant images by applying one of the various filters available in our application, much like Instagram does. Once properly implemented, the processed image will then be available to export as an image file for use in social media, advertisements, etc.

\subsection{The Stakeholders}

Stakeholders include: social media users (Instagram, Snapchat), photo/video editors, Team NAR (us), Spencer Smith (Professor) and future developers who may take over the project if it requires more man-power.

\subsubsection{The Client}

Everyday photo takers wishing to transform their images into something more elegant by applying photo filters. 

\subsubsection{The Customers}

Users of social media outlets such as Instagram or Snapchat. This can be a platform to further build on those applications by adding new filters and implementing unique video processing filters that neither application currently supports. 

\subsubsection{Other Stakeholders}


Contributors: Future and current developers interested in the development process of this project and are willing to share their knowledge in order to further improve the image processing framework.\\

Testers: This may include developers of our application, or general users of the program who will test the projects functionality and usability.\\

Image Processing Specialist: Developing new in-depth image processing algorithms and possibly video processing algorithms as well. Our team will require an expert in the field in order to make future upgrades possible. The addition of a specialist will really allow our application to include some unique processing powers that will take us to the next level of image computing and allow us to compete with the likes of Instagram and Snapchat.


\subsection{Mandated Constraints}
This section provides information about the mandated constraints we face in our project.
\subsubsection{Solution Constraints}
\begin{itemize}
  \item {\bf Description:} Project must be completed by the week of November 14th, 2016\\
        {\bf Rationale:} Date that revision 0 demonstration is due\\
        {\bf Fit Criterion:} N/A \\
  \item {\bf Description:} The Java application must be functional on all operating systems\\
        {\bf Rationale:} Users should not be limited to only using Windows or just 1 operating\\ system but include as many as possible allowing the product to potentially reach as many users as possible\\
        {\bf Fit Criterion:} Testers and users will ensure that it works for all operating systems
  \end{itemize}
  
\subsubsection{Implementation Environment of the Current System}

This project will be coded in Java, using the Marvin Image Processing Framework.

\subsubsection{Partner of Collaborative Applications}

N/A

\subsubsection{Off-the-Shelf Software}

There are many implementations of Marvin frameworks in Java, which we plan on using as a reference to ensure we are following the standard coding style.

\subsubsection{Anticipated Workplace Environment}

We plan to make our application available on any computer system (Window, Mac, Linux).

\subsubsection{Schedule Constraints}

\begin{itemize}
  \item Test Plan Revision 0: October 28th, 2016
  \item Proof of Concept Demonstration: Week of October 17th, 2016
  \item Design Document Revision 0: November 11th, 2016
  \item Revision 0 Demonstration: Week of November 14th, 2016
  \item Final Demonstration: Week of November 28th, 2016
  \item Final Documentation: December 7th, 2016
\end{itemize}

\subsubsection{Budget Constraints}
Currently there is no budget constraint for this project, it will be open source and free for use when created. Later into the product lifecycle we may look at monetization.

\subsection{Naming Conventions and Terminology}

{\bf Image Filter:} A software routine that changes the appearance of an image or part of an image by altering the shades and colours of the pixels in some manner. Filters are used to increase brightness and contrast as well as to add a wide variety of textures, tones and special effects to a picture.\\

{\bf Image Processing:} The analysis and manipulation of a digitized image, especially in order to improve its quality.\\

{\bf Framework:} In computer programming, a framework is an abstraction in which common code providing generic functionality can be selectively overridden or specialized by user code providing specific functionality.


\subsection{Relevant Facts and Assumptions}

\begin{itemize}
  \item Users will have a basic understanding of how to work a computer
  \item The user's desktop can support a Java application
\end{itemize}

\section{Functional Requirements}

This section provides information about functional requirements implemented in our project.
\subsection{The Scope of the Work and the Product}

Our image processing app is designed to provide a reliable, user-friendly, and simplistic approach to editing ones photos. Through the use of a vast library of Marvin Frameworks, we are able to offer the user with many options on how they want to edit their photos. The app enables the user to quickly access common tasks through an interactive interface. By using graphical utilities such as buttons to access image processing functions, users can efficiently explore and converge on a solution for a particular image processing problem.

\clearpage

\subsubsection{The Context of the Work}

refer to \textit{Figure 1}

\begin{figure}[h]
	\includegraphics[width=\linewidth]{context.jpg}
	\caption{Context of the Work}
	\end{figure}
	
Explanation:

\begin{itemize}
	\item The external entity is the user who initially selects the image they would like to process. The image then goes through the external image processing interface which calls the Marvin framework to apply the users inputted desired filter. The interface then displays a preview of the image and this cycle continues until the user exports the file into a raster image file they see suitable.
\end{itemize}

\subsubsection{Work Partitioning}
refer to \textit{Table 2}

\begin{table}[bp]
\caption{\bf Work Partitioning}
\begin{tabu} to 1.1\textwidth { | X[l] | X[1] | X[1] | }

\hline
	      		Event & Input/Output & Summary \\
\hline
1. Application prompts user to select file & none & Application asks user to select an image of file type png/jpg from their local drive.  \\
\hline
2. User selects image & File from local drive of type png/jpg & User chooses desired image to edit through interface. \\
\hline
3. Application uploads selected image to main interface & none & Upon selection, users desired file is uploaded onto main interface. \\
\hline
4. User selects desired processing option & JButton selected, MarvinPluginLoader. loadImagePlugin(desired option) & User chooses one of the options on the main interface and their choice is processed. \\
\hline
5. Application processes image & Preview of the processed image shows up on main interface & Based on the user's choice, the application selects the appropriate plugin and processes the image, then displaying it on the main interface.\\
	     \hline
6. Application prompts user to export image & none & Through an external interface the user is prompted to save their processed image to their local drive.\\
	     \hline
7. User selects destination and name of file & Processed image stored to local drive & The user may choose where to store the processed image as well as what to name it. Once completed file is saved to local drive.\\
	     \hline
	   
	\end{tabu}
	\end{table}

	
	\newpage



\subsubsection{Individual Product Use Cases}
refer to \textit{Figure 2}

. . .

\begin{figure}[h]
	\includegraphics[width=\linewidth]{uses.png}
	\caption{Uses Case Diagram}
	\end{figure}


\subsection{Functional Requirements}

	\textbf{Requirement \# 1}\\
	\textbf{Requirement Type:  9}\\
	\textbf{Event/use case: 1}\\
	\textbf{Description:} The application must be able to accept an image of file type png or jpg.\\
	\textbf{Rationale:} Images come in different formats and sizes, and to truly be use friendly more than one file type should be accepted.\\
	\textbf{Fit Criterion:} Application can accept and process pngs and jpgs.

	\noindent\rule{12cm}{0.4pt} \\ 

	\noindent\textbf{Requirement \# 2}\\
	\textbf{Requirement Type: 9}\\
	\textbf{Event/use case: 2,3,4}\\
	\textbf{Description:} The application must be able to select correct plugin corresponding to selected processing option.\\
	\textbf{Rationale:} There are many processing options that the user can choose from, and the program should be able to recognize the chosen one, and process the image accordingly. \\
	\textbf{Fit Criterion:} The image is processed correctly.

	\noindent\rule{12cm}{0.4pt} \\

	\noindent\textbf{Requirement \# 3}\\
	\textbf{Requirement Type: 9}\\
	\textbf{Event/use case: 2,3,4}\\
	\textbf{Description:} The application must display a preview of the selected processing option upon selection.\\
	\textbf{Rationale:} Allows users to see how the option they selected effects their chosen image.\\
	\textbf{Fit Criterion:} The image is processed correctly.\\

	\noindent\rule{12cm}{0.4pt} \\

	\noindent\textbf{Requirement \# 4}\\
	\textbf{Requirement Type: 9}\\
	\textbf{Event/use case: 5}\\
	\textbf{Description:} Once the user is satisfied with their edit, the application must prompt them with an external interface, asking them to save it to their local drive. \\
	\textbf{Rationale:}  Most users will be editing their image with a purpose in mind, maybe they want to post it on social media or use it on their personal website. Regardless, having an option to export would fulfill this need.\\
	\textbf{Fit Criterion:} User is prompted to export image.\\

	\noindent\rule{12cm}{0.4pt} \\
	
	\newpage

	\noindent\textbf{Requirement \# 5}\\
	\textbf{Requirement Type: 9}\\
	\textbf{Event/use case: n/a}\\
	\textbf{Description:} The application must respond appropriately to the closing, minimizing, maximizing, as well as resizing functions of the window.\\
	\textbf{Rationale:} These are basic functions that most users are already familiar with and that the application should follow too.\\
	\textbf{Fit Criterion:}Application closes when the "X" button is pressed, minimizes once "-" button is pressed, maximized when the "+" button is pressed, and resizes in the direction of the mouse drag.\\ \\ \\

\newpage

\section{Non-functional Requirements}
This section provides information about non-functional requirements implemented in our project.

\subsection{Look and Feel Requirements}

\begin{itemize}
\item This application's user interface will be visually appealing and will operate seamlessly on any operating system.
\item The application will take a new approach on the way users edit and filter their photos, they will be able to achieve new and better results to attain the perfect image. 
\end{itemize}

\subsection{Usability and Humanity Requirements}

\begin{itemize}
\item The software will have a simple UI that will allow for any individual no matter their age or familiarity with image processing to be able to use the application.
\item There will be no manual or guide on how to use the application, therefore The UI must be extremely intuitive to result in a plane learning curve.
\item The software will use a standard coding style and will be well documented for anyone interested in examining the code.
\end{itemize}

\subsection{Performance Requirements}

Speed
\begin{itemize}
\item The app should take no longer than PROCESSING\_TIME to load.
\item Users will be able to upload an image in a maximum of UPLOAD\_TIME.
\item All interaction between a user and the UI will be instantaneous. Response times should not exceed RESPONSE\_TIME to complete any action. This includes all the different filters that are available.
\end{itemize}
Reliability
\begin{itemize}
\item For now we plan to release on desktop computer as this is a java application. Due to the nature of not actually needing a wireless connection nor connection to any server, the user can use this app freely, and whenever they see fit.
\item After taking the appropriate measures to accommodate for any risks that may arise during the use of the application, the program should not crash but instead send an error message if misused.
\end{itemize}
Quality
\begin{itemize}
\item Through no point in the image processing (including addition of filters and saving the new image) will the resolution be affected from its original value nor will any form of distortion occur.
\end{itemize}
Capacity
\begin{itemize}
\item Due to the local nature of the program and its independency from any sever there is no limit to the amount of people that may be processing there images.
\item Storage is based solely on the limitations of each individual user. The application can support processing of STORAGE amount of photos.
\end{itemize}


\subsection{Operational and Environmental Requirements}

Physical Environment
\begin{itemize}
\item The application will be available to anyone wishing to process an image.
\end{itemize}
Technological Environment
\begin{itemize}
\item The software will run on any desktop computer and is supported on all operating systems.
\item At the moment any image size will be accepted as long as the file format is either a PNG or a JPG and will be returned as a PNG or JPG.
\item All app data and images will be stored within each users local hard drive or any other personal storage device.
\end{itemize}

\subsection{Maintainability and Support Requirements}

\begin{itemize}
\item The program will be continuously updated to fix any bugs and add more functionality.
\item New patches will be released whenever we have made any major updates or fixes so users can experience the best software we have to offer.
\item Proper documentation will be maintained as the software expands and new code is added.
\item To broaden the scope of the project we plan to start accepting more file types as well as introduce video processing capabilities.
\item The program will perform on any desktop but will also be portable, meaning it can be stored on a USB or external hard drive and run from there.
\end{itemize}

\subsection{Security Requirements}

\begin{itemize}
\item Using correct segregation of the design, we can achieve information hiding thus protecting us from potential intellectual property theft.
\end{itemize}

\subsection{Cultural Requirements}

\begin{itemize}
\item Filters will not resemble any offensive, profane or non-ethical ideologies.
\item The application will include filters which parallel current pop culture.
\end{itemize}

\subsection{Legal Requirements}

\begin{itemize}
\item Throughout the documentation we reference the use of The Marvin Framework. Our citation legalizes the use of their plugins.
\item No legal documentation is needed in terms of the images themselves as the application is only responsible for the processing of images not the supply of them.
\item Include am EULA to outline how users can use the software as well as to protect from fraud and piracy.
\end{itemize}

\subsection{Health and Safety Requirements}
\begin{itemize}
	\item  The occurrence of an eye strain from using the application for too long could be possible issue, if this occurs take a minimum 15 minute break from using digital screens but if symptoms still persist contact your doctor.
\end{itemize}


\newpage

\section{Project Issues}

This section provides information about the issues we face in our project.

\subsection{Open Issues}
At the current stage of development there are no issues to be considered.

\subsection{Off-the-Shelf Solutions}

There are many implementations of Marvin frameworks in Java, which we plan on using as a reference to ensure we are following the standard coding style.

\subsection{New Problems}

Currently we are using the Marvin framework image processing pre-set filters and implementing them for a better user experience so no new problems should arise. We are also considering adding video processing at this point but it may be out of the scope for this project due to our time constraint, it would be a lot harder to implement resulting in more natural occurring problems.

\subsection{Tasks}

\begin{itemize}
\item Requirements document revision 0
\item Test plan revision 0
\item Proof of concept demonstration
\item Design document revision 0
\item Revision 0 demonstration
\item Test report revision 0
\item Final Demonstration revision 1  
\item Write final revisions to document
\end{itemize}

\subsection{Migration to the New Product}

N/A

\subsection{Risks}

There are no legal risks our products can violate due to the fact we do not store any personal information. The only risk our application can encounter is if our code is not efficient and uses too much ram when processing images and can cause the computer to overheat but no serious risk is applicable to our application.

\subsection{Costs}

There are no costs associated with the development of the project because we are redeveloping an open source project. If we plan to further development and add more features we could possibly need a bigger team but for now there will be no direct costs.

\subsection{User Documentation and Training}

For our project we plan to optimize the usability and implement a UI that is as straight forward as possible allowing any age group to be able to use our application. For running our executable JAR file we have added a readme file to explain the steps to process your first image.

\subsection{Waiting Room}

\begin{itemize}	
\item Future releases my include buttons in different languages
\item Our application could possibly implement video filter processing
\item In the near future we plan to add the ability to use your webcam in combination with the application
\item We plan to implement a web interface similar to Instagram
\end{itemize}

\subsection{Ideas for Solutions}

\begin{itemize}
\item Users of the application will be able to download the the program in the language of their choice
\item Create a functionality to allow video files to be processed. With the current time constraint this may not be plausible but with more time or more developers this can be a future addition to the application.
\end{itemize}

\section{Appendix}

This section has been added to the Volere template.  This is where you can place
additional information.

\subsection{Symbolic Parameters}

\begin{itemize}
\item PROCESSING\_TIME = 3 Seconds
\item UPLOAD\_TIME = 2 Seconds
\item RESPONSE\_TIME = 1/2 a Second
\item STORAGE = Disk Space of User
\end{itemize}

\end{document}